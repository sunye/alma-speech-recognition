\documentclass[a4paper,11pt]{report}

\usepackage{alma}

\makeindex
\dominitoc

\begin{document}


\begin{titlepage}

\vspace*{0cm}



\begin{flushleft}
	\hspace{1cm} \includegraphics*[width=4cm]{images/logo.jpg}\\
	\hspace{1cm} \textsl{Université de Nantes}\\
	\hspace{1cm} \textsl{12, rue de la Houssinière}\\
	\hspace{1cm} \textit{44322 Nantes}
	\hrulefill
\end{flushleft}




\vspace{2cm}

\begin{flushright}

	{\fontfamily{ppl}\fontseries{b}\fontsize{1.4cm}{1.65cm}\selectfont 
Alma Speech Recognition} 	 \\
	{\fontfamily{ppl}\fontseries{b}\fontsize{0.7cm}{0.825cm}\selectfont 
Projet de fin d'études} 	 \\

	
	\vspace{1cm}
	

	
	\vspace{1cm}
	Jérémy \textsc{Braud} \\
	Gaëtan \textsc{Hervouet} \\
	Cédric \textsc{Krommenhoek} \\
	Damien \textsc{Lévin} \\
	\textit{2009-2010}
	
\end{flushright}


\vspace{0cm}

\begin{flushleft}



	\hspace{1cm} \textsc{Master 2 - ALMA}\\
	
\end{flushleft}

\hspace*{0,5cm}\hrulefill
\end{titlepage}
% \hspace{\stretch{1}}


\begin{abstract}
Le projet de fin d'études consiste en la mise en \oe{}uvre de compétences acquises au cours des deux années de Master.
Chaque groupe de quatre à six étudiants travaille sur un sujet de projet différent.
Le déroulement du projet doit respecter les étapes de création d'un véritable projet open-source et doit en conséquent tirer au maximum profit des différents outils de gestion de projet.

Le projet que nous avons choisi de réaliser a été proposé par l'entreprise IBM~France\footnote{IBM~France -- \url{http://www.ibm.com/fr/fr/}}.
Il s'agit du fruit d'un partenariat entre IBM et l'Université de Nantes\footnote{La convention de partenariat a été signée le 24 novembre 2009.} afin de faciliter la scolarisation des étudiants malentendants.
En effet, à l'aide d'un logiciel disposant d'un moteur de reconnaissance vocale (en l'occurrence le moteur ``Speechroot'' d'IBM), l'étudiant aurait accès au sous-titrage du discours de l'enseignant de manière immédiate.
Ce projet diffère légèrement des autres par le fait que nous devons travailler en collaboration avec des groupes d'étudiants de d'autres écoles.
L'équipe d'IBM, composée de Béatrice~Martin et de sa collègue Christel~Amato, technicienne ayant une très importante expérience dans le domaine de la reconnaissance vocale, se charge de la coordination entre les trois groupes d'étudiants (celui de Centrale\footnote{Centrale Nantes -- \url{http://www.ec-nantes.fr/}}, de Polytech'\footnote{Polytech'Nantes -- \url{http://www.polytech.univ-nantes.fr/}} et nous-mêmes) et Stéphane~Brunat du Relai Handicap.
\end{abstract}

\tableofcontents


\chapter*{Introduction}
L'objectif de notre projet de fin d'études consiste à réaliser une application dans un temps imparti, en utilisant toutes nos compétences en matière de qualité logicielle et de gestion de projet.
Nous devons répondre aux exigences d'un client, M.~Brunat, pour que notre solution puisse être mise en place par l'Université.

Le choix des technologies à employer est libre, mais il nous faut im\-pé\-ra\-ti\-ve\-ment avoir quelque part dans notre architecture un code Java pouvant dialoguer avec le moteur Speechroot.
Ce moteur a été implémenté en C et nous n'avons pas l'autorisation de consulter ses sources, mais l'environnement JNI\footnote{Java Native Interface} permettant l'accès à ce code en java nous est fourni.


\include{Besoins}
\chapter{Architectures envisagés}

\section{Préambule}
Le cahier des charges présenté ci-dessus spécifie que l'application doit être
simple et mobile. Malgré certaines recommandations par le client, nous avons
tout de même étudié les choix qui s'offrait à nous en partant de l'existant.

\section{Système OrKestre}

Le système OrKestre \ref{fig:orkestre} est un système actuellement sur le marché. Il est aussi
connu sous le nom de système 'Perroquet'.


\begin{figure}[h]
 \centering
 \includegraphics[scale=0.5]{./img/orKestre.png}
 % homePanel.png: 0x0 pixel, -2147483648dpi, 0.00x0.00 cm, bb=
 \caption{Architectures - Système OrKestre}
 \label{fig:orkestre}
\end{figure}

Il se présente ainsi : Une personne est présente dans le cours et dicte distinctement le cours qu'elle entends du professeur, dans un microphone. Ce microphone est relié à un ordinateur qui dispose d'un moteur de reconnaissance vocale. Une fois le flux audio transformé en texte, ce dernier est envoyé à l'utilisateur malentendant par le réseau wifi ou filaire.

Ce système a pour avantage d'être optimal dans la qualité du texte reconnu par le moteur de reconnaissance vocale. En effet, le dicteur maîtrise l'outil et y possède son dictionnaire ainsi que ses modèles vocaux. 
Pour autant, ce système n'est pas utilisable à l'université de Nantes car il nécessite l'engagement d'une personne tierce entièrement consacrée à la diction dans le microphone. Ceci est un problème pour l'université qui ne dispose pas de fond suffisant pour l'emploi et la formation d'une, voire de plusieurs personnes pour ce système.

\section{Solution 1}


\begin{figure}[h]
 \centering
 \includegraphics[scale=0.5]{./img/solution1.png}
 % homePanel.png: 0x0 pixel, -2147483648dpi, 0.00x0.00 cm, bb=
 \caption{Architectures - Solution 1}
 \label{fig:solution1}
\end{figure}



La première solution envisagée (\ref{fig:solution1}) suit donc le modèle précédent mais sans une personne tierce.
Le microphone est donc directement relié à un ordinateur portable géré par le professeur.
Le son est transmit par cette ordinateur à un serveur fonctionnant sous Windows XP (requis pour faire tourner Speechroot) qui analyse le texte le transmet à un client web sur l'ordinateur de l'utilisateur malentendant.

\subsection{Avantages}
Cette solution est la plus simple pour l'utilisateur. En effet, elle ne contraint pas celui-ci à utiliser un système d'exploitation tel que Windows Xp puisque le client web est compatible avec tout système qui fournit un navigateur web. Le texte serait donc affiché sur un site web interne à l'université.

\subsection{Inconvénients}
Le problème est que cette solution n'est pas envisageable car certain professeurs sont ré\-frac\-tai\-res à l'informatique. Le professeur ne doit donc pas avoir à manipuler un ordinateur mais juste à porter le microphone.


\section{Solution 2}

La solution 2 (\ref{fig:solution2}) reprends donc la solution précédente, mais on supprime l'ordinateur manipulé par le professeur.

\begin{figure}[h]
 \centering
 \includegraphics[scale=0.5]{./img/solution2.png}
 % homePanel.png: 0x0 pixel, -2147483648dpi, 0.00x0.00 cm, bb=
 \caption{Architecture - Solution 2}
 \label{fig:solution2}
\end{figure}

Le microphone perds son fil et est donc un récepteur est connecté à l'ordinateur de l'élève.
C'est donc par son ordinateur que le flux audio est transmis au serveur de l'université qui fait fonctionner le moteur de reconnaissance vocale.

Le texte lui est renvoyé, comme auparavant, dans un client web.

\subsection{Avantages}
Le professeur n'a pas à se soucier de l'ordinateur portable : l'élève connecte le micro du professeur à son ordinateur personnel.

\subsection{Inconvénients}
Cette solution n'est tout de même pas envisageable à l'université. Bien qu'elle soit la plus simple et la plus mobile pour le professeur et l'elève, l'université refuse de maintenir un serveur fonctionnant sur Windows XP.

\section{Solution 3}
La solution 3 (\ref{fig:solution3}) est finalement celle que nous avons choisi d'implémenter car c'est la seule qui est réalisable même si on perds en mobilité, utilisabilité et simplicité.


\begin{figure}[h]
 \centering
 \includegraphics[scale=0.5]{./img/solution3.png}
 % homePanel.png: 0x0 pixel, -2147483648dpi, 0.00x0.00 cm, bb=
 \caption{Architecture - Solution 3}
 \label{fig:solution3}
\end{figure}


En effet, le moteur de reconnaissance vocale est finalement déplacé sur l'ordinateur personnel de l'étudiant : il est exécuté en tache de fond sur celui-ci. L'ordinateur de l'étudiant exécute donc un logiciel qui est uniquement compatible avec Windows. On perds donc en portabilité mais seule cette solution est réalisable.


\chapter{Développement de l'application}
\minitoc

\section{Intégration du moteur de reconnaissance vocale}

\subsection{Préambule}
La reconnaissance vocale de notre application s'effectue à l'aide du moteur Speechroot, développé il y a une dizaine d'années par l'entreprise IBM en langage C.
Pour des raisons de confidentialité, nous n'avons pas accès au code source de ce moteur mais à son interface qui nous ouvre à son exploitation.
Cette interface JNI va nous permettre de l'intégrer dans notre application, que nous écrivons en Java.
Comme nous l'avons déjà signalé auparavant, ce moteur a été compilé pour Windows et ne peut donc pas être exécuté sur d'autres systèmes d'exploitation.

\subsection{Fonctionnalités du moteur}
A partir de l'interface JNI, nous avons accès à une liste de fonctionnalités que nous pouvons alors appeler.
Les fonctions de bases du moteur sont les suivantes~:
\begin{itemize}
\item démarrage et arrêt du moteur de reconnaissance vocale.
\item ouverture et fermeture du microphone.
\item ouverture de l'interface de gestion des dictionnaires.
\item ouverture de l'interface de gestion des modèles vocaux.
\end{itemize}	

Le moteur effectue ses retours à l'aide d'une fonction de callback que nous lui spécifions lors d'une initialisation obligatoire, avant son démarrage.
Cette fonction prend en paramètres deux chaînes de caractères représentant le type de message et son corps.
Les retours qui nous intéressent le plus particulièrement sont donc ceux qui correspondent à ce qui a été traduit par le moteur.
Ces retours sont identifiés par le type de message \texttt{onNewReco} et leur corps respecte la syntaxe~:
\begin{verbatim}
WORDS###confidence score###Pronunciation###begin word time//end word time
    ###flags
\end{verbatim}
où \texttt{WORDS} correspond aux mots reconnus et le reste aux paramètres du moteur.
Les paramètres correspondent à~:
\begin{description}
\item [\texttt{confidence score}~:] nombre compris entre -100 et 100 et qui représente au degré de confiance de la reconnaissance.
\item [\texttt{Pronunciation}~:] la prononciation des mots, similaire à l'alphabet phonétique international.
\item [\texttt{begin word time}~:] l'heure de début de la reconnaissance de ces mots.
\item [\texttt{end word time}~:] l'heure de fin de la reconnaissance de ces mots.
\item [\texttt{flags}~:] Drapeaux indiquant des informations supplémentaires telles que :
\begin{itemize}
\item le prochain mot commencera par une majuscule.
\item le prochain mot sera collé au précédent.
\item etc.
\end{itemize}
\end{description}

Le moteur offre également d'autres fonctionnalités que nous n'exploiterons pas, comme par exemple la reconnaissance vocale d'un fichier audio ou la possibilité de conserver le flux audio de la transcription.

\subsection{Intégration}
Afin que notre application soit au maximum évolutive et qu'elle ne dépende pas d'un unique moteur de reconnaissance vocale, nous avons conçu une interface disposant des fonctionnalités de base que nous avons énuméré auparavant (voir figure~\ref{fig:engineDiagram}).
Comme notre objectif est d'intégrer en particulier le moteur Speechroot, nous avons pensé cette interface de manière à ce qu'elle se calque parfaitement avec lui.

\begin{figure}[ht!]
 \centering
 \includegraphics[scale=.5,keepaspectratio=true]{./images/EngineDiagram.png}
 % EngineDiagram.png: 621x294 pixel, 72dpi, 21.90x10.37 cm, bb=0 0 621 294
 \caption{Diagramme des classes du moteur de reconnaissance vocale}
 \label{fig:engineDiagram}
\end{figure}

La présence de la classe abstraite implémentant l'interface se justifie par le fait que le moteur est observé par le contrôleur de notre application.
Il doit en conséquent étendre la classe \texttt{java.util.Observable}, ce que nous ne pouvons pas spécifier avec seulement une interface.

Dans le cas de Speechroot, le moteur doit utiliser une classe implémentant l'interface fournie par IBM~: \texttt{TreatMessageInterface}.
Cette interface spécifie la fonction de callback qui doit être passée en paramètre lors du démarrage du moteur.
C'est donc dans l'implémentation de cette fonction que nous décodons le corps du message pour ensuite le transmettre au contrôleur à l'aide du patron de conception ``observateur''.

Nous avons rencontré quelques difficultés concernant les retours de callback du moteur Speechroot.
En effet, les retours ne s'effectuaient pas correctement~: nous ne les récupérions pas tant que nous n'exécutions pas un appel aux interfaces de gestion des dictionnaires ou des modèles vocaux.
Nous en avons déduis qu'il devait probablement y avoir un processus bloquant au sein du moteur.
Pour cette raison, nous avons écris rapidement un bouchon simulant l'action de ce moteur, afin de pouvoir poursuivre notre projet sans nous soucier de ce problème que nous ne pouvions pas régler seuls.
Puisque que tous ces moteurs étendent la même classe, et que c'est cette classe abstraite que nous appelons dans le contrôleur, il est extrêmement facile d'échanger le bouchon par le véritable moteur Speechroot, dès que le soucis de ce dernier sera réglé.
Bien que ce n'était pas notre intention originelle, nous avons préféré lors de notre soutenance faire la démonstration de notre application munie du bouchon plutôt que du moteur Speechroot pour n'avoir que des résultats fonctionnels.



\section{Déploiement de la base de données embarquée}



\chapter{Interface homme machine}

\section{Fonctionnalités de l' IHM}

L'interface homme-machine est un des composant essentiel de notre projet. En effet, c'est grâce à elle que l'utilisateur va pouvoir juger de la qualité du logiciel. Ainsi, conformément au  cahier des charges, on va retrouver les fonctionnalités suivantes :

\begin{itemize}
 \item Explorateur de documents
\begin{itemize}
 \item Module
 \item Dossier
 \item Cours
\end{itemize} 
 \item Visualisation du cours
 \item Editeur de texte
\begin{itemize}
 \item Outils de formatage
 \item Affichage du plan 
\end{itemize} 
 \item Export au format PDF
 \item Export/import au format RTF (Format Microsoft)
 \item Impression du cours édité

\end{itemize}



\section{Ergonomie}

\subsection{Solutions de visualisation et d'édition envisagées}

En ce qui concerne l'ergonomie du logiciel plusieurs choix de conception ont été mis en confrontation. Le premier, qui n'a finalement pas été retenu était d'utiliser un unique éditeur de texte. Celui-ci aurai joué un  double rôle, tout d'abord celui d'un panneau de visualisation du cours, prononcé par l'enseignant. Le cours aurai été affiché à la fin du document. De plus, l'utilisateur aurai eu la possibilité de modifier le panneau de visualisation dans le but d'organiser son  cours.

Le principal soucis de cette solution était que le travail de restructuration du cours en direct aurai été une tâche particulièrement fastidieuse et aurai fortement perturber la suivie du cours.

Ainsi, la solution finalement envisagée est l'utilisation de deux panneaux. Un premier dont le but est d'afficher le texte brut issu du moteur de reconnaissance vocale. Celui-ci à pour but de permettre à l'utilisateur mal entendant d'utiliser sa  vision en remplacement. Le second panneau lui à pour but de servir de support de cours, offrant la possibilité de prise de notes. De la même façon qu'un étudiant normal prendrai des notes sur papier.

Bien que cette solution semble la plus productive et la plus intéressante, elle est n'est pas exempt de défauts. Ainsi, il faut bien souligner le fait qu'il est particulièrement difficile de lire un texte tout en réalisant des  prises de notes.     



\subsection{Accessibilité}

Un des points fondamentaux de l'IHM était de concevoir une interface facile d'utilisation et un maximum intuitive. Ainsi, un effort particulier à été réalisé de façon à ce que l'utilisateur prenne rapidement en main l'outil. Pour cela différents points ont étés réalisés :

\begin{itemize}
 \item Une barre d'outils avec des icônes explicites : pour permettre un accès rapide au principales fonctions
 \item Création de raccourcis claviers classiques (CTRL-C, CTRL-V ...)
 \item Utilisation des standards d'ergonomie : Placement des menus, des boutons ...
\end{itemize}



\section{Conception MVC}


\begin{figure}[h]
 \centering
 \includegraphics[scale=0.6]{./images/homePanel.png}
 % homePanel.png: 0x0 pixel, -2147483648dpi, 0.00x0.00 cm, bb=
 \caption{IHM - Panneau d'accueil}
 \label{fig:homePanel}
\end{figure}



\begin{figure}[h]
 \centering
 \includegraphics[scale=0.6]{./images/homePanel.png}
 % homePanel.png: 0x0 pixel, -2147483648dpi, 0.00x0.00 cm, bb=
 \caption{IHM - Panneau de travail}
 \label{fig:homePanel}
\end{figure}

\chapter{Outils de mis en \oe{}uvre}
\minitoc

Ce chapitre détaille les outils utilisés tout au long de ce projet

\section{Communication}

Pour assurer la bonne marche du projet, nous avions un panel d’outils de communication.
Nous avons eu la possibilité, grâce à notre partenariat avec IBM, d’utiliser Lotus QuickR.
Il nous a essentiellement servi à communiquer entre les différentes écoles, afin que chacun puisse progresser.
Nous avons aussi créé un groupe Google, permettant la diffusion d’informations à une liste de personnes spécifiées.
Durant ce projet, nous avons pu découvrir un nouveau site, Basecamp.
Basecamp est un service de gestion de projets en ligne.
Parmi les services proposés, nous avons surtout utilisé le système d’affectation de tâches.
Cela nous aura permis de mieux nous organiser, et de voir l’avancement des différentes tâches assignées, à la manière de Scrum\footnote{méthode agile pour la gestion de projets}.

\section{Gestion des sources}

Pour pouvir gérer efficacement le code source de notre programm, nous avons d’abord mis en place un dépôt Subversion sur un serveur Google.
C’est un logiciel de gestion de versions, qui permet de ne plus s’occuper des différentes modifications de chacun, car il les agrège automatiquement.
Cela nous a permis d’être plus productifs, surtout grâce aux versions intégrées à nos éditeurs favoris, à savoir Eclipse et Netbeans.


Nous utilisons aussi pour le code source, conjointement à Subversion, un outil nommé Maven.
Maven est essentiellement un outil de gestion et de compréhension de projet.
Il offre, par le biais de son fichier de description « pom.xml »,  des fonctionnalités de :
\begin{itemize}
\item Construction, compilation
\item Documentation
\item Rapport
\item Gestion des bibliothèques et de leurs dépendances
\item Gestion des sources
\item Mise à jour de projet
\item Déploiement
\end{itemize}

C’est un outil minimaliste et à la fois très puissant qui permet aussi l’adjonction de "plugins", afin de réaliser d’autres fonctions, comme la réalisation du site web associé au projet ou les tests unitaires par exemple.
Voici les principaux plugins usités :
\begin{itemize}
\item Tests (JUnit)
\item Persistance (persistence, hibernate, H2…)
\item Outils d’analyse de code
	\begin{itemize}
	\item Checkstyle
	\item Cobertura 
	\item PMD
	\end{itemize}
\item Création du site web
\item Rapport d’activité
\item Documentation (Javadoc)
\end{itemize}

\section{IHM}

Afin de réaliser l’IHM (Interface Homme Machine), ou plus précisément les bases de celle-ci, nous avons utilisé des outils de développement WYSIWYG\footnote{What You See Is What You Get} intégrés aux éditeurs, à savoir Jigloo et Netbeans Swing GUI Builder\footnote{outils respectivement pour Eclipse et Netbeans}.
Bien que ces outils soient assez complets, il nous aura fallu tout de même savoir maîtriser Swing et AWT, les bibliothèques graphiques utilisées.

\chapter{Qualité Logicielle}

Ce chapitre détaille les aspects de qualité logicielle qui ont été pris en compte tout le long du projet.

\section{Sonar}

60 révisions 
3640 lignes de codes 
Couverture de test de 100% sur la base de données
(IHM non testable) 
Respect des règles de style « checkstyle » (Convention syntaxique)
Utilisation de Sonar
 Evaluation du code 
 Détection des problèmes d’interdépendance 
 Calcul de la complexite


\chapter{Bilan}

Dans les deux mois de travail imparti, au rythme d'une journée par semaine, nous avons réussi à implémenter la plupart des fonctionnalités que nous souhaitions développer. 
Nous n'avons cependant pas la satisfaction d'avoir correctement fait fonctionner Speechroot avec notre interface homme-machine, alors que celle-ci est entièrement prête à communiquer avec le moteur de reconnaissance vocal d'IBM.
En effet, des soucis dans le fonctionnement de l'interface JNI de Speechroot nous ont empêchés d'atteindre cet objectif.

Voici donc la liste des fonctionnalités de l'application~:

\section{Fonctionnel}
\begin{itemize}
	\item Stockage des cours dans la base de données,
	\item Organisation des cours dans les dossiers,
	\item Panneau d'accueil,
	\item Édition de texte,
	\item Export en RTF et PDF,
	\item Affichage du texte dicté (via un bouchon).
\end{itemize}

\section{Non-fonctionnel}
\begin{itemize}
	\item Communication avec SpeechRoot,
	\item Impression du texte,
	\item Import de différents formats.	
\end{itemize}


\chapter{Conclusion}

Ce projet de fin d'études à été l'occasion de développer une application dont le bénéfice est extrêmement important pour les élèves malentendants.
La collaboration avec IBM et les autres école de l'Université a permis au projet de voir le jour pour atteindre cet objectif.
Bien que nous ayons dialogué avec IBM, les échanges avec les autres écoles ont été difficiles.
En effet, pour les trois écoles, ce projet n'avait pas les mêmes contraintes~: durées différentes, dates de commencement du projet différentes, etc.
Nous avons cependant pu rencontrer les élèves de Polytech' et confronter nos avis sur le projet.
Cela nous a permis de nous diriger vers ce qui nous semblait la meilleure solution au niveau de l'ergonomie, et de nous créer un espace d'échanges adapté au développement pour le partage de l'avancement du projet.
\medskip
Cela nous a également permis de mettre en application les dernières notions vues au cours du Master.
Ainsi, en plus des classiques étapes d'analyse, conception et développement que nous avons pour habitude de respecter, nous avons pu appliquer des méthodes de qualité logicielle tout au long du projet.
En effet, ce projet de fin d'étude a été mené de façon à atteindre nos objectifs, bien évidemment, mais en nous aidant d'outils de gestion de projet, de gestion de code et de communication ainsi que des méthodes étudiées au cours de notre Master.
\medskip
Il constitue finalement un pas en avant vers les méthodes de travail qui sont utilisées en entreprise où la collaboration et la qualité priment.


\listoffigures   

\appendix

\chapter{Suivi de projet}

\begin{description}
\item[14 janvier~:] Premier contact avec l'équipe d'IBM, réunion de démarrage du projet au LINA\footnote{Laboratoire d'Informatique de Nantes-Atlantique} avec Mme~Martin, Mme~Amato et M.~Sunyé~; aperçu de la problématique et des besoins.
\item[19 janvier (matin)~:] Réunion au LINA avec M.~Sunyé~; mise en place du dépôt Subversion.
\item[19 janvier (après-midi)~:] Premier contact avec M.~Brunat du Relai Handicap~; spécifications des besoins de l'utilisateur de l'outil, précisions sur les contraintes techniques.
\end{description}


\chapter{Diagramme de Gantt}

Bien que nous ne nous soyons pas vraiment attribués de fonctions telles que développeur, architecte ou même chef de projet, nous nous sommes répartis une grande partie des tâches à accomplir pour l'avancement du projet.
En conséquent, nous obtenons le diagramme de Gantt de la figure~\ref{}. %TODO


\printindex


\end{document}
