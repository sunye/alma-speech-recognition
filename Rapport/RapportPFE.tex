\documentclass[a4paper,11pt]{report}

\usepackage[utf8]{inputenc}
\usepackage[frenchb]{babel}
\usepackage[pdftex]{graphicx}
\usepackage{makeidx}

\usepackage[french]{minitoc}
\usepackage{hyperref}

\title{Rapport du projet de fin d'études\\
	\huge{\textsc{Alma Speech Recognition}}}
\author{Jérémy \textsc{Braud} \and Gaëtan \textsc{Hervouet} \and Cédric \textsc{Krommenhoek} \and Damien \textsc{Lévin}}

\makeindex
\dominitoc

\begin{document}

\maketitle

\begin{abstract}
Le projet de fin d'études consiste en la mise en \oe{}uvre de compétences acquises au cours des deux années de Master.
Chaque groupe de quatre à six étudiants travaille sur un sujet de projet différent.
Le déroulement du projet doit respecter les étapes de création d'un véritable projet open-source et doit en conséquent tirer au maximum profit des différents outils de gestion de projet.

Le projet que nous avons choisi de réaliser a été proposé par l'entreprise IBM France\footnote{IBM France -- \url{http://www.ibm.com/fr/fr/}}.
Il s'agit du fruit d'un partenariat entre IBM et l'Université de Nantes afin de faciliter la scolarisation des étudiants malentendants.
En effet, à l'aide d'un logiciel disposant d'un moteur de reconnaissance vocale (en l'occurence le moteur Speechroot d'IBM), l'étudiant aurait accès au sous-titrage du discours de l'enseignant de manière immédiate.
\end{abstract}

\tableofcontents

\chapter*{Introduction}


\end{document}
