\documentclass[a4paper,11pt]{report}

\usepackage{alma}

\usepackage[utf8x]{inputenc}
\usepackage[frenchb]{babel}
\usepackage[pdftex]{graphicx}
\usepackage{makeidx}

\usepackage[french]{minitoc}
\usepackage{url}


\makeindex

\begin{document}


\begin{titlepage}

\vspace*{0cm}



\begin{flushleft}
	\hspace{1cm} \includegraphics*[width=4cm]{images/logo.jpg}\\
	\hspace{1cm} \textsl{Université de Nantes}\\
	\hspace{1cm} \textsl{12, rue de la Houssinière}\\
	\hspace{1cm} \textit{44322 Nantes}
	\hrulefill
\end{flushleft}




\vspace{2cm}

\begin{flushright}

	{\fontfamily{ppl}\fontseries{b}\fontsize{1.4cm}{1.65cm}\selectfont 
Alma Speech Recognition} 	 \\
	{\fontfamily{ppl}\fontseries{b}\fontsize{0.7cm}{0.825cm}\selectfont 
Projet de fin d'études} 	 \\

	
	\vspace{1cm}
	

	
	\vspace{1cm}
	Jérémy \textsc{Braud} \\
	Gaëtan \textsc{Hervouet} \\
	Cédric \textsc{Krommenhoek} \\
	Damien \textsc{Lévin} \\
	\textit{2009-2010}
	
\end{flushright}


\vspace{0cm}

\begin{flushleft}



	\hspace{1cm} \textsc{Master 2 - ALMA}\\
	
\end{flushleft}

\hspace*{0,5cm}\hrulefill
\end{titlepage}
% \hspace{\stretch{1}}


\begin{abstract}
Le projet de fin d'études consiste en la mise en \oe{}uvre de compétences acquises au cours des deux années de Master.
Chaque groupe de quatre à six étudiants travaille sur un sujet de projet différent.
Le déroulement du projet doit respecter les étapes de création d'un véritable projet open-source et doit en conséquent tirer au maximum profit des différents outils de gestion de projet.

Le projet que nous avons choisi de réaliser a été proposé par l'entreprise IBM~France\footnote{IBM~France -- \url{http://www.ibm.com/fr/fr/}}.
Il s'agit du fruit d'un partenariat entre IBM et l'Université de Nantes\footnote{La convention de partenariat a été signée le 24 novembre 2009.} afin de faciliter la scolarisation des étudiants malentendants.
En effet, à l'aide d'un logiciel disposant d'un moteur de reconnaissance vocale (en l'occurence le moteur ``Speechroot'' d'IBM), l'étudiant aurait accès au sous-titrage du discours de l'enseignant de manière immédiate.
Ce projet diffère légèrement des autres par le fait que nous devons travailler en collaboration avec des groupes d'étudiants de d'autres écoles.
L'équipe d'IBM, composée de Béatrice~Martin et de sa collègue Christel~Amato, technicienne ayant une très importante expérience dans le domaine de la reconnaissance vocale, se charge de la coordination entre les trois groupes d'étudiants (celui de Centrale\footnote{Centrale Nantes -- \url{http://www.ec-nantes.fr/}}, de Polytech'\footnote{Polytech'Nantes -- \url{http://www.polytech.univ-nantes.fr/}} et nous-mêmes) et Stéphane~Brunat du Relai Handicap.
\end{abstract}

\tableofcontents


\chapter*{Introduction}
L'objectif de notre projet de fin d'études consiste à réaliser une application dans un temps imparti, en utilisant toutes nos compétences en matière de qualité logicielle et de gestion de projet.
Nous devons répondre aux exigences d'un client, M.~Brunat, pour que notre solution puisse être mise en place par l'Université.

Le choix des technologies à employer est libre, mais il nous faut im\-pé\-ra\-ti\-ve\-ment avoir quelque part dans notre architecture un code Java pouvant dialoguer avec le moteur Speechroot.
Ce moteur a été implémenté en C et nous n'avons pas l'autorisation de consulter ses sources, mais l'environnement JNI\footnote{Java Native Interface} permettant l'accès à ce code en java nous est fourni.


\chapter{Identification des besoins}

L'outil que nous développons est à destination d'étudiants malentendants.
Le profil de ces utilisateurs varie donc de l'utilisateur débutant à l'utilisateur avancé.
Pour être exploitable, notre logiciel doit donc être simple en même temps que fonctionnel, et la prise en main doit être immédiate.

Pour que la reconnaissance vocale soit efficace, il faut absolument que le microphone employé soit de très haute qualité.
Le choix du microphone a été soumis à Mme Amato afin de profiter de son expérience dans ce domaine.
Il faudra tout de même effectuer quelques tests pour s'assurer que le microphone est bien adapté à l'usage qu'on souhaite en faire.

Dans l'idéal, nous aimerions que l'outil ne soit pas limité à certaines plateformes afin que l'étudiant puisse conserver la sienne.
Nous sommes conscients que le choix du moteur Speechroot va contraindre l'utilisateur à choisir Windows comme système d'exploitation, mais cette limitation peut toutefois être contournée par l'utilisation d'une machine virtuelle sur l'ordinateur de l'étudiant, à condition que celui-ci soit suffisamment performant.
Cette solution reste néanmoins complexe et n'est pas à la portée de n'importe quel utilisateur.
Il sera donc vivement conseillé aux étudiants novices en la matière de disposer de Windows.
\chapter{Architectures envisagés}

\section{Préambule}
Le cahier des charges présenté ci-dessus spécifie que l'application doit être
simple et mobile. Malgré certaines recommandations par le client, nous avons
tout de même étudié les choix qui s'offrait à nous en partant de l'existant.

\section{Système OrKestre}

Le système OrKestre \ref{fig:orkestre} est un système actuellement sur le marché. Il est aussi
connu sous le nom de système 'Perroquet'.


\begin{figure}[h]
 \centering
 \includegraphics[scale=0.5]{./img/orKestre.png}
 % homePanel.png: 0x0 pixel, -2147483648dpi, 0.00x0.00 cm, bb=
 \caption{Architectures - Système OrKestre}
 \label{fig:orkestre}
\end{figure}

Il se présente ainsi : Une personne est présente dans le cours et dicte distinctement le cours qu'elle entends du professeur, dans un microphone. Ce microphone est relié à un ordinateur qui dispose d'un moteur de reconnaissance vocale. Une fois le flux audio transformé en texte, ce dernier est envoyé à l'utilisateur malentendant par le réseau wifi ou filaire.

Ce système a pour avantage d'être optimal dans la qualité du texte reconnu par le moteur de reconnaissance vocale. En effet, le dicteur maîtrise l'outil et y possède son dictionnaire ainsi que ses modèles vocaux. 
Pour autant, ce système n'est pas utilisable à l'université de Nantes car il nécessite l'engagement d'une personne tierce entièrement consacrée à la diction dans le microphone. Ceci est un problème pour l'université qui ne dispose pas de fond suffisant pour l'emploi et la formation d'une, voire de plusieurs personnes pour ce système.

\section{Solution 1}


\begin{figure}[h]
 \centering
 \includegraphics[scale=0.5]{./img/solution1.png}
 % homePanel.png: 0x0 pixel, -2147483648dpi, 0.00x0.00 cm, bb=
 \caption{Architectures - Solution 1}
 \label{fig:solution1}
\end{figure}



La première solution envisagée (\ref{fig:solution1}) suit donc le modèle précédent mais sans une personne tierce.
Le microphone est donc directement relié à un ordinateur portable géré par le professeur.
Le son est transmit par cette ordinateur à un serveur fonctionnant sous Windows XP (requis pour faire tourner Speechroot) qui analyse le texte le transmet à un client web sur l'ordinateur de l'utilisateur malentendant.

\subsection{Avantages}
Cette solution est la plus simple pour l'utilisateur. En effet, elle ne contraint pas celui-ci à utiliser un système d'exploitation tel que Windows Xp puisque le client web est compatible avec tout système qui fournit un navigateur web. Le texte serait donc affiché sur un site web interne à l'université.

\subsection{Inconvénients}
Le problème est que cette solution n'est pas envisageable car certain professeurs sont ré\-frac\-tai\-res à l'informatique. Le professeur ne doit donc pas avoir à manipuler un ordinateur mais juste à porter le microphone.


\section{Solution 2}

La solution 2 (\ref{fig:solution2}) reprends donc la solution précédente, mais on supprime l'ordinateur manipulé par le professeur.

\begin{figure}[h]
 \centering
 \includegraphics[scale=0.5]{./img/solution2.png}
 % homePanel.png: 0x0 pixel, -2147483648dpi, 0.00x0.00 cm, bb=
 \caption{Architecture - Solution 2}
 \label{fig:solution2}
\end{figure}

Le microphone perds son fil et est donc un récepteur est connecté à l'ordinateur de l'élève.
C'est donc par son ordinateur que le flux audio est transmis au serveur de l'université qui fait fonctionner le moteur de reconnaissance vocale.

Le texte lui est renvoyé, comme auparavant, dans un client web.

\subsection{Avantages}
Le professeur n'a pas à se soucier de l'ordinateur portable : l'élève connecte le micro du professeur à son ordinateur personnel.

\subsection{Inconvénients}
Cette solution n'est tout de même pas envisageable à l'université. Bien qu'elle soit la plus simple et la plus mobile pour le professeur et l'elève, l'université refuse de maintenir un serveur fonctionnant sur Windows XP.

\section{Solution 3}
La solution 3 (\ref{fig:solution3}) est finalement celle que nous avons choisi d'implémenter car c'est la seule qui est réalisable même si on perds en mobilité, utilisabilité et simplicité.


\begin{figure}[h]
 \centering
 \includegraphics[scale=0.5]{./img/solution3.png}
 % homePanel.png: 0x0 pixel, -2147483648dpi, 0.00x0.00 cm, bb=
 \caption{Architecture - Solution 3}
 \label{fig:solution3}
\end{figure}


En effet, le moteur de reconnaissance vocale est finalement déplacé sur l'ordinateur personnel de l'étudiant : il est exécuté en tache de fond sur celui-ci. L'ordinateur de l'étudiant exécute donc un logiciel qui est uniquement compatible avec Windows. On perds donc en portabilité mais seule cette solution est réalisable.



\chapter{Interface homme machine}

\section{Fonctionnalités de l' IHM}

L'interface homme-machine est un des composant essentiel de notre projet. En effet, c'est grâce à elle que l'utilisateur va pouvoir juger de la qualité du logiciel. Ainsi, conformément au  cahier des charges, on va retrouver les fonctionnalités suivantes :

\begin{itemize}
 \item Explorateur de documents
\begin{itemize}
 \item Module
 \item Dossier
 \item Cours
\end{itemize} 
 \item Visualisation du cours
 \item Editeur de texte
\begin{itemize}
 \item Outils de formatage
 \item Affichage du plan 
\end{itemize} 
 \item Export au format PDF
 \item Export/import au format RTF (Format Microsoft)
 \item Impression du cours édité

\end{itemize}



\section{Ergonomie}

\subsection{Solutions de visualisation et d'édition envisagées}

En ce qui concerne l'ergonomie du logiciel plusieurs choix de conception ont été mis en confrontation. Le premier, qui n'a finalement pas été retenu était d'utiliser un unique éditeur de texte. Celui-ci aurai joué un  double rôle, tout d'abord celui d'un panneau de visualisation du cours, prononcé par l'enseignant. Le cours aurai été affiché à la fin du document. De plus, l'utilisateur aurai eu la possibilité de modifier le panneau de visualisation dans le but d'organiser son  cours.

Le principal soucis de cette solution était que le travail de restructuration du cours en direct aurai été une tâche particulièrement fastidieuse et aurai fortement perturber la suivie du cours.

Ainsi, la solution finalement envisagée est l'utilisation de deux panneaux. Un premier dont le but est d'afficher le texte brut issu du moteur de reconnaissance vocale. Celui-ci à pour but de permettre à l'utilisateur mal entendant d'utiliser sa  vision en remplacement. Le second panneau lui à pour but de servir de support de cours, offrant la possibilité de prise de notes. De la même façon qu'un étudiant normal prendrai des notes sur papier.

Bien que cette solution semble la plus productive et la plus intéressante, elle est n'est pas exempt de défauts. Ainsi, il faut bien souligner le fait qu'il est particulièrement difficile de lire un texte tout en réalisant des  prises de notes.     



\subsection{Accessibilité}

Un des points fondamentaux de l'IHM était de concevoir une interface facile d'utilisation et un maximum intuitive. Ainsi, un effort particulier à été réalisé de façon à ce que l'utilisateur prenne rapidement en main l'outil. Pour cela différents points ont étés réalisés :

\begin{itemize}
 \item Une barre d'outils avec des icônes explicites : pour permettre un accès rapide au principales fonctions
 \item Création de raccourcis claviers classiques (CTRL-C, CTRL-V ...)
 \item Utilisation des standards d'ergonomie : Placement des menus, des boutons ...
\end{itemize}



\section{Conception MVC}


\begin{figure}[h]
 \centering
 \includegraphics[scale=0.6]{./images/homePanel.png}
 % homePanel.png: 0x0 pixel, -2147483648dpi, 0.00x0.00 cm, bb=
 \caption{IHM - Panneau d'accueil}
 \label{fig:homePanel}
\end{figure}



\begin{figure}[h]
 \centering
 \includegraphics[scale=0.6]{./images/homePanel.png}
 % homePanel.png: 0x0 pixel, -2147483648dpi, 0.00x0.00 cm, bb=
 \caption{IHM - Panneau de travail}
 \label{fig:homePanel}
\end{figure}

\chapter{Qualité Logicielle}

Ce chapitre détaille les aspects de qualité logicielle qui ont été pris en compte tout le long du projet.

\section{Sonar}

60 révisions 
3640 lignes de codes 
Couverture de test de 100% sur la base de données
(IHM non testable) 
Respect des règles de style « checkstyle » (Convention syntaxique)
Utilisation de Sonar
 Evaluation du code 
 Détection des problèmes d’interdépendance 
 Calcul de la complexite

\chapter{Bilan}

\chapter{Conclusion}

\appendix

\chapter{Suivi de projet}

\begin{description}
\item[14 janvier~:] Premier contact avec l'équipe d'IBM, réunion de démarrage du projet au LINA\footnote{Laboratoire d'Informatique de Nantes-Atlantique} avec Mme~Martin, Mme~Amato et M.~Sunyé~; aperçu de la problématique et des besoins.
\item[19 janvier (matin)~:] Réunion au LINA avec M.~Sunyé~; mise en place du dépôt Subversion.
\item[19 janvier (après-midi)~:] Premier contact avec M.~Brunat du Relai Handicap~; spécifications des besoins de l'utilisateur de l'outil, précisions sur les contraintes techniques.
\end{description}


\chapter{Diagramme de Gantt}

Bien que nous ne nous soyons pas vraiment attribués de fonctions telles que développeur, architecte ou même chef de projet, nous nous sommes répartis une grande partie des tâches à accomplir pour l'avancement du projet.
En conséquent, nous obtenons le diagramme de Gantt de la figure~\ref{}. %TODO


\printindex


\end{document}
