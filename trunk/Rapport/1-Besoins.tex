\chapter{Identification des besoins}

L'outil que nous développons est à destination d'étudiants malentendants.
Le profil de ces utilisateurs varie donc de l'utilisateur débutant à l'utilisateur avancé.
Pour être exploitable, notre logiciel doit donc être simple en même temps que fonctionnel, et la prise en main doit être immédiate.

Pour que la reconnaissance vocale soit efficace, il faut absolument que le microphone employé soit de très haute qualité.
Le choix du microphone a été soumis à Mme Amato afin de profiter de son expérience dans ce domaine.
Il faudra tout de même effectuer quelques tests pour s'assurer que le microphone est bien adapté à l'usage qu'on souhaite en faire.

Dans l'idéal, nous aimerions que l'outil ne soit pas limité à certaines plateformes afin que l'étudiant puisse conserver la sienne.
Nous sommes conscients que le choix du moteur Speechroot va contraindre l'utilisateur à choisir Windows comme système d'exploitation, mais cette limitation peut toutefois être contournée par l'utilisation d'une machine virtuelle sur l'ordinateur de l'étudiant, à condition que celui-ci soit suffisamment performant.
Cette solution reste néanmoins complexe et n'est pas à la portée de n'importe quel utilisateur.
Il sera donc vivement conseillé aux étudiants novices en la matière de disposer de Windows.