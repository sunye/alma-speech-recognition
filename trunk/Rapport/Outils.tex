\chapter{Outils de mis en \oe{}uvre}
\minitoc

Ce chapitre détaille les outils utilisés tout au long de ce projet

\section{Communication}

Pour assurer la bonne marche du projet, nous avions un panel d’outils de communication.
Nous avons eu la possibilité, grâce à notre partenariat avec IBM, d’utiliser Lotus QuickR.
Il nous a essentiellement servi à communiquer entre les différentes écoles, afin que chacun puisse progresser.
Nous avons aussi créé un groupe Google, permettant la diffusion d’informations à une liste de personnes spécifiées.
Durant ce projet, nous avons pu découvrir un nouveau site, Basecamp.
Basecamp est un service de gestion de projets en ligne.
Parmi les services proposés, nous avons surtout utilisé le système d’affectation de tâches.
Cela nous aura permis de mieux nous organiser, et de voir l’avancement des différentes tâches assignées, à la manière de Scrum\footnote{méthode agile pour la gestion de projets}.

\section{Gestion des sources}

Pour pouvir gérer efficacement le code source de notre programm, nous avons d’abord mis en place un dépôt Subversion sur un serveur Google.
C’est un logiciel de gestion de versions, qui permet de ne plus s’occuper des différentes modifications de chacun, car il les agrège automatiquement.
Cela nous a permis d’être plus productifs, surtout grâce aux versions intégrées à nos éditeurs favoris, à savoir Eclipse et Netbeans.


Nous utilisons aussi pour le code source, conjointement à Subversion, un outil nommé Maven.
Maven est essentiellement un outil de gestion et de compréhension de projet.
Il offre, par le biais de son fichier de description « pom.xml »,  des fonctionnalités de :
\begin{itemize}
\item Construction, compilation
\item Documentation
\item Rapport
\item Gestion des bibliothèques et de leurs dépendances
\item Gestion des sources
\item Mise à jour de projet
\item Déploiement
\end{itemize}

C’est un outil minimaliste et à la fois très puissant qui permet aussi l’adjonction de "plugins", afin de réaliser d’autres fonctions, comme la réalisation du site web associé au projet ou les tests unitaires par exemple.
Voici les principaux plugins usités :
\begin{itemize}
\item Tests (JUnit)
\item Persistance (persistence, hibernate, H2…)
\item Outils d’analyse de code
	\begin{itemize}
	\item Checkstyle
	\item Cobertura 
	\item PMD
	\end{itemize}
\item Création du site web
\item Rapport d’activité
\item Documentation (Javadoc)
\end{itemize}

\section{IHM}

Afin de réaliser l’IHM (Interface Homme Machine), ou plus précisément les bases de celle-ci, nous avons utilisé des outils de développement WYSIWYG\footnote{What You See Is What You Get} intégrés aux éditeurs, à savoir Jigloo et Netbeans Swing GUI Builder\footnote{outils respectivement pour Eclipse et Netbeans}.
Bien que ces outils soient assez complets, il nous aura fallu tout de même savoir maîtriser Swing et AWT, les bibliothèques graphiques utilisées.
